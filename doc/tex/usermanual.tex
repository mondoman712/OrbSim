\documentclass[a4paper,11pt,titlepage]{article}
\usepackage{graphicx,
	float,
	hyperref}

\DeclareGraphicsExtensions{.png, .eps}

\pagestyle{fancy}
\author{Sam Smith}
\title{OrbSim User Manual}
\date{ }
\begin{document}
\maketitle
\tableofcontents
\clearpage

\section{Installation}
OrbSim is distributed as an executable file. To use it, download the file and
double click it.

\section{Using OrbSim}

After starting the program, you should see this (the actual appearance may vary
from system to system).
\begin{figure}[H]
	\includegraphics[width=\textwidth]{../img/start.png}
\end{figure}

The simulation should be running already, if it isn't try closing the program
and opening it again.

\subsection{Adding a body to the simulation}
To add a body to the simulation, just click the submit button in the menu. A new
body should appear in the system and start  orbiting the sun. The supplied
values are good values for a simple elliptical orbit. The parameters in the add
body section of the menu can be used to change where the body starts, how fast
it goes and how it looks. A higher pos-x value will make the body start further
to the right. A higher pos-y value will make the body start further above the
sun. A higher vel-x value will make the body start moving faster towards the
right of the screen, and a higher vel-y value will make the body start moving
faster towards the top of the screen. The size value is the radius of the body
in pixels and the id is the bodies identifier, which is used in the list in the
remove body section.

\subsection{Removing a body from the simulation}
To remove a body from the system, select the body to remove and click the remove
button. If no body is selected an error message will appear to ask you to select
a body.

\subsection{Saving the current simulation state}
To save the system, click the save button near the bottom of the menu. It is
advised that you change the filename first because bodies.txt is loaded by
default, and stores the default system.

\subsection{Loading a saved simulation state}
To load a system, enter the name of the file you wish to load and click load.

\section{Common Errors}
\subsection{Invalid integer when adding bodies}
When adding a body to the simulation, this error may occur.
\begin{figure}[H]
	\includegraphics[width=\textwidth]{../img/add2.png}
\end{figure}
To prevent it, make sure that the position and velocity values are all between
10000 and 10000000000, and the size is below 100. If it continues to happen try
closing the program and opening it again.

\subsection{No body selected to remove}
When trying to remove a body, this error message may appear.
\begin{figure}[H]
	\includegraphics[width=\textwidth]{../img/rm3.png}
\end{figure}

If it does, click OK and try selecting the body you would like to remove again.
If it continues to happen try adding a new body and removing it, if it still
continues try closing the program and opening it.

\subsection{Invalid file name while saving}

\subsection{File does not exist}
While attempting to load a saved system

\end{document}
