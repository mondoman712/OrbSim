%!TEX root = ../project.tex
\part{Maintenance}

\section{System Overview}

\section{Code}



\subsection{Code Listing}
\lstinputlisting{../src/main.lisp}

\subsection{Function Design}

\paragraph{draw-bodies}
The draw-bodies function is used to call the draw-body function on a list of
bodies. This allows the bodies to be kept in a list and outputted easily from
that list.

\begin{pseudocode}{draw-bodies}{bodies}
	\IF bodies
	  \THEN 
	    \BEGIN
		$draw first item in list of bodies$ \\
		$call draw-bodies on rest of list$
	    \END
	  \ELSE
      	    \RETURN{True}	
\end{pseudocode}

\begin{lstlisting}
(defun draw-bodies (bodies)
  (if bodies
      (progn (draw-body (car bodies))
	     (draw-bodies (cdr bodies)))
      `t)) 
\end{lstlisting}

\paragraph{update-vel}
This function is used to update the velocity of a body.

\begin{pseudocode}{update-vel}{body}
	\BEGIN
	accel \GETS \CALL{split-force}{\CALL{calc-g}{sun, body}, 
					\CALL{ang}{sun, body}} \\
	bodyvel \GETS bodyvel + accel
	\END
\end{pseudocode}

\begin{lstlisting}
(defun update-vel (body)
  (let ((accel (split-force 
                 (calc-g (mass *sun*) (dist (pos body) (pos *sun*)))
                 (ang (pos *sun*) (pos body)))))
    (sets #'+ (x (vel body)) (x accel))
    (sets #'+ (y (vel body)) (y accel))))
\end{lstlisting}

\section{Development Environment}

\section{External Libraries used}
\subsection{SDL}
I used SDL (Simple DirectMedia Layer) to produce graphics for my program, so SDL
is required for my program to run. I used a library called "Lispbuilder-SDL" to
allow me to interface with SDL from Lisp, this is not required for the program
to run as it compiles itself into the executable.
