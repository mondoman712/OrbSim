%!TEX root = ../project.tex
\part{Maintenance}

\section{System Overview}

\section{Code}



\subsection{Code Listing}
\lstinputlisting{../src/main.lisp}

\subsection{Function Design}

\paragraph{draw-bodies}
The draw-bodies function is used to call the draw-body function on a list of
bodies. This allows the bodies to be kept in a list and outputted easily from
that list.
\begin{pseudocode}{draw-bodies}{bodies}
	\IF bodies
	\THEN \BEGIN
		draw first item in list of bodies
		call draw-bodies on rest of list
		\END
	\ELSE
	
\end{pseudocode}



\section{Development Environment}

\section{External Libraries used}
\subsection{SDL}
I used SDL (Simple DirectMedia Layer) to produce graphics for my program, so SDL
is required for my program to run. I used a library called "Lispbuilder-SDL" to
allow me to interface with SDL from Lisp, this is not required for the program
to run as it compiles itself into the executable.
