%!TEX root = ../project.tex
\part{Maintenance}

\section{System Overview}


\section{Code Listing}
\subsection{main.lisp}
\lstinputlisting{../src/main.lisp}
\subsection{io.lisp}
\lstinputlisting{../src/io.lisp}
\subsection{menu.lisp}
\lstinputlisting{../src/menu.lisp}


\section{Function Design}

\subsection{draw-bodies}
The draw-bodies function is used to call the draw-body function on a list of
bodies. This allows the bodies to be kept in a list and outputted easily from
that list. It does this by calling draw-body on the first item in the list, and
then calling itself on all the items but the first.

\begin{pseudocode}{draw-bodies}{bodies}
	\IF bodies
	  \THEN 
	    \BEGIN
	    	\CALL{draw-body}{$first item in bodies$} \\
		\CALL{draw-bodies}{$rest of bodies$}
	    \END
	  \ELSE
      	    \RETURN{\TRUE}	
\end{pseudocode}

\begin{lstlisting}
(defun draw-bodies (bodies)
  (if bodies
      (progn (draw-body (car bodies))
	     (draw-bodies (cdr bodies)))
      `t)) 
\end{lstlisting}

\subsection{update-vel}
This function is used to update the velocity of a body. The pseudocode is
simplified to make it easier to understand, I've just removed some of the
function calls for $accel$ and grouped the $x$ and $y$ values. 

\begin{pseudocode}{update-vel}{body}
	\BEGIN
	accel \GETS \CALL{split-force}{\CALL{calc-g}{sun, body}, 
					\CALL{ang}{sun, body}} \\
	bodyvel \GETS bodyvel + accel
	\END
\end{pseudocode}

\begin{lstlisting}
(defun update-vel (body)
  (let ((accel (split-force 
                 (calc-g (mass *sun*) (dist (pos body) (pos (car *bodies*)))
                 (ang (pos (car *bodies*)) (pos body)))))
    (sets #`+ (x (vel body)) (x accel))
    (sets #`+ (y (vel body)) (y accel))))
\end{lstlisting}

\subsection{read-bodies}
This is used to read through the list of lists in a file and turn each list
within the main list into a body in the system. \\
\begin{pseudocode}{read-bodies}{filename}
	\FOREACH bod $ in $ \CALL{read-list}{filename}
	\DO $collect $ \CALL{make-body}{bod} $ into $ bods\\
	\RETURN{bods}
\end{pseudocode}

\begin{lstlisting}
(defun read-bodies (filename)
  (loop for bod in (read-list filename)
       collecting (make-instance 
		   `body
		   :pos (make-instance
			 `point
			 :x (nth 0 bod)
			 :y (nth 1 bod))
		   :vel (make-instance
			 `point
			 :x (nth 2 bod)	
			 :y (nth 3 bod))
		   :mass (nth 4 bod)
		   :size (nth 5 bod)
		   :colour (sdl:color :r (nth 6 bod)
				      :g (nth 7 bod)
				      :b (nth 8 bod)))))
\end{lstlisting}

\subsection{remove-nth}
This is used to remove the nth item from a list, which I am using to remove a
body from the system. It does this recursively, by creating a list containing
the first item in the list and the result of calling itself on $n - 1$ and all
the items but the first in the list.\\

\begin{pseudocode}{remove-nth}{n, list}
	\PROCEDURE{car}{list}
	\RETURN{$First item in list$}
	\ENDPROCEDURE

	\PROCEDURE{cdr}{list}
	\RETURN{$All items but first in list$}
	\ENDPROCEDURE

	\PROCEDURE{cons}{items}
	\RETURN{$A list containing the given arguments$}
	\ENDPROCEDURE

	\MAIN
	\IF n == 0 \OR list $ is empty$ 
	\THEN \RETURN{$ all items but first in $ list} 
	\ELSE
	\RETURN{\CALL{make-list}{\CALL{first-item}{list},
			\CALL{remove-nth}{n-1, \CALL{cdr}{list}}}} 
	\ENDMAIN
\end{pseudocode}
\begin{lstlisting}
(defun remove-nth (n list)
  (if (or (zerop n) (null list))
      (cdr list)
      (cons (car list) (remove-nth (1- n) (cdr list)))))
\end{lstlisting}

\section{Variable and Function Listings}
\subsection{Global Variable List}
\begin{tabular}{p{0.2\textwidth}p{0.2\textwidth}p{0.6\textwidth}}
	Name & Type & Description \\ \hline
	G & Number & Stores the gravitational constant.\\
	screen-size & Point & Stores the screen size of the SDL window. \\
	bodies & list & Stores all of the bodies in the system \\

\end{tabular}

\subsection{Function List --- main.lisp}
\begin{tabular}{p{0.2\textwidth}p{0.3\textwidth}p{0.5\textwidth}}
	Name & Arguments & Description\\ \hline
	make-body & pos-x, pos-y, vel-x, vel-y, mass, size, colour, id &
		Makes the process of creating instances of the body class a
		little easier\\
	pos2pos & pos & Converts position relative to centre to sdl coordinates
	\\
	calc-g & M, r & Calculates the acceleration due to gravity \\
	dist & a, b & Calculates the distance between two points \\
	ang & a, b & Calculates the angle between two points \\
	split-force & mag, ang & Splits a force into X and Y components \\
	sets (macro) & fn, a, b & sets the variable a to the result of fn called
	with a and b \\
	update-vel & body & Updates the velocity of a body \\
	update-pos & body & Updates the position of a body \\
	update & body & Calls update-vel and update-pos on a body \\
	update-list & bodies & Calls update on a list of bodies \\
	draw-body & body & Draws a body to the screen \\
	draw-bodies & bodies & Calls draw-body on a list of bodies \\
	col & colour & Returns the colour given as an SDL colour, if no valid
	colour is given a random one is returned \\
	add-body & pos-x, pos-y, vel-x, vel-y, mass, size, colour, id & Adds a
	body to the system \\
	rm-body & id & Removes a body from the system \\
	sdl-init & \emph{none} & Initialises the SDL Environment \\
	sdl-main-loop & \emph{none} & The main SDL Loop \\
	main & \emph{none} & Starts the SDL portion of the program in a thread,
	and the menu in another
\end{tabular}

\subsection{Function List --- io.lisp}
\begin{tabular}{p{0.2\textwidth}p{0.3\textwidth}p{0.5\textwidth}}
	Name & Arguments & Description\\ \hline
	body-to-list & body & Converts an instance of the body class into a list
	of values \\ 
	bodies-to-list & bodies & Calls body-to-list on a list of bodies \\
	save-list & lst, filename & Opens a file and prints a list into it \\
	save-bodies & filename & Calls save-list and bodies-to-list on the
	global variable bodies \\
	read-list & filename & Reads a list from a file \\
	read-bodies & filename & Reads a list from a file and converts it into a
	list of bodies \\ 
\end{tabular}

\subsection{Function List --- menu.lisp}
\begin{tabular}{p{0.2\textwidth}p{0.3\textwidth}p{0.5\textwidth}}
	Name & Arguments & Description\\ \hline
	parse-int & str & Turns a string into an integer, and makes sure its
	valid \\
	remove-nth &n, list & removes the nth value from a list \\
	listbox-update & listbox & Updates the listbox in the menu when called
	\\
	error-message & message, (button-text) & Creates a message box with the
	message given, and the button text can also be changed. \\
	main & \emph{none}  & Creates the menu \\
\end{tabular}

\section{Development Environment}
Orbsim has developed using, and has been tested with SBCL (Steel Bank Common
Lisp). It will only run in SBCL due to the fact that I've used SBCL's threading
features and not all Lisp Implementations support threading, and those that do
implement it differently. The "Lispbuilder-SDL" package is required, and it is
advised that QuickLisp is used to install it (if not the start of "main.lisp"
will need to be changed). 

\section{External Libraries used}
\subsection{SDL}
I used SDL (Simple DirectMedia Layer) to produce graphics for my program, so SDL
is required for my program to run. I used a library called "Lispbuilder-SDL" to
allow me to interface with SDL from Lisp, this is not required for the program
to run as it compiles itself into the executable.

\subsection{Tk}
I used Tk as it allowed me to easily create a graphical menu system. I used the
LTk library to interface with Tk from Lisp.
