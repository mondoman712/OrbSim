\documentclass[a4paper,11pt,titlepage]{article}
\usepackage{graphicx,
       	fancyhdr,
       	tikz,
       	tabularx,
       	pseudocode,
       	hyperref,
      	amsmath,
	listings,
	float,
	pdfpages
	}

\DeclareGraphicsExtensions{.png, .eps}
\pagestyle{fancy}
\setlength{\headheight}{15pt}
\newcounter{tmpc}

\renewenvironment{description}[1][0pt]
  {\list{}{\labelwidth=0pt \leftmargin=#1
   \let\makelabel\descriptionlabel}}
  {\endlist}

\renewcommand{\arraystretch}{1.3}

\lheadme{Sam Smith}
\rhead{Center Name: King Edward VI Five Ways School}

\author{Sam Smith}
\title{Orbit Simulator}
\date{May? 2014}
\begin{document}
\maketitle
\tableofcontents
\clearpage

\definecolor{darkgreen}{rgb}{0,0.6,0}

\lstset{language=lisp,
	basicstyle=\scriptsize\ttfamily,
	numbers=left,
	numberstyle=\tiny,
	numbersep=9pt,
        extendedchars=true,      
        breaklines=true,        
        showspaces=false,      
        showtabs=false,       
        xleftmargin=17pt,
        framexleftmargin=17pt,
        framexrightmargin=5pt,
        framexbottommargin=4pt, showstringspaces=false,
	keywordstyle=\color{blue},
	commentstyle=\color{gray},
	stringstyle=\color{darkgreen},
	keepspaces=true,
	columns=flexible,
	escapeinside={#\"}{#}
	}
\newcommand{\add}[1]{\input{./tex/#1.tex} \clearpage}

\add{analysis}
\add{design}
\add{testing}
\add{implementation}
\add{maintenance}
\part{User Manual}
The user manual is a document that can be given to the user to explain the
program to them and how to use it. It will cover everything that they will need
so that they can be given just this and the program and then they can do
anything they need with it. It will cover all of the main functions of the
program, the different states of the program and common errors that the user may
encounter.
\includepdf[pages=-]{./tex/usermanual.pdf}
\add{evaluation}

\end{document}
