\documentclass[a4paper,11pt,titlepage]{article}

\author{Sam Smith}
\title{My A2 Computing Project}

\begin{document}
\maketitle
\tableofcontents
\part*{Analysis}
\addcontentsline{toc}{part}{Analysis}

\begin{description}
	\item[What is your idea?] \hfill \\
		A program to help visualize the orbits of planets and satellites
	\item[Is your idea complex enough?] \hfill \\
		It will require a lot of complex equations to calculate the
		positions of each object at any given time, and the use of
		graphics libraries to display the objects.
	\item[Who is your user?] \hfill \\
		My end user is Dr.~Naylor (dnaylor@kefw.org). I have not yet
		decided whether to aim it towards use in front of a class, or
		for the students to use themselves.
	\item[What programming languages do you know?] \hfill \\
		My main language is C, but I also know some Python and I am
		currently learning some Common Lisp. I plan to write the program
		mainly in C because of its good performance and compatibility
		with many external libraries. I might try to include some Lisp
		in as well.
	\item[Do you need to use a database or other data stored?] \hfill \\
		I don't think I will require any large databasing tool like SQL,
		but I might have external files for storing information about
		planets.
	\item[What do you need to learn?] \hfill \\
		I will need to learn how to use the external libraries that I
		intend to use for graphics. I plan on using either OpenGL or
		SDL.
	\item[What websites and books can help you?] \hfill \\
		Wikipedia will be very useful for finding figures like the mass
		of a planet. I have a copy of \emph{The C Programming Language}
		by Dennis Ritchie and Brian Kernighan, which I use as a
		reference while I am programming in C.
		\\
\end{description}

\section{Investigation}
\subsection{Background}

King Edward VI Five Ways school teaches Astronomy at GCSE level. One part of the
coarse involves understanding the orbits of planets. It is very hard to
visualize without some sort of animation to help. At the moment they use a basic
web applet, with minimal features.

\subsection{Problem}

\subsection{Description of the Current System}

\end{document}
